% Created 2016-07-09 Sat 17:20
\documentclass[11pt]{article}
\usepackage[utf8]{inputenc}
\usepackage[T1]{fontenc}
\usepackage{fixltx2e}
\usepackage{graphicx}
\usepackage{grffile}
\usepackage{longtable}
\usepackage{wrapfig}
\usepackage{rotating}
\usepackage[normalem]{ulem}
\usepackage{amsmath}
\usepackage{textcomp}
\usepackage{amssymb}
\usepackage{capt-of}
\usepackage{hyperref}
\usepackage{braket}
\DeclareMathOperator{\ima}{Im}
\author{evan}
\date{\today}
\title{MA353 - Linear Algebra II}
\hypersetup{
 pdfauthor={evan},
 pdftitle={MA353 - Linear Algebra II},
 pdfkeywords={},
 pdfsubject={},
 pdfcreator={Emacs 24.5.1 (Org mode 8.3.4)}, 
 pdflang={English}}
\begin{document}

\maketitle
\tableofcontents


\section{Vectorspaces and Subspaces}
\label{sec:orgheadline3}
\begin{definition}
\textbf{vectorspace}

A set of vectors V is a vectorspace if:

\begin{itemize}
\item closed under addition and scalar multiplication - \(\overrightarrow{v} + \overrightarrow{w} \in V\), \(c \overrightarrow{v} \in V\)
\item an identity is defined - \(1 \cdot \overrightarrow{v} = \overrightarrow{v}\)
\end{itemize}
\end{definition}

\subsection{Properties of Vectorspaces}
\label{sec:orgheadline1}
\begin{itemize}
\item commutative - \(\overrightarrow{v} + \overrightarrow{w} = \overrightarrow{w} + \overrightarrow{v}\)
\item distributive - \((a + b)\overrightarrow{v} = a\overrightarrow{v} + b \overrightarrow{v}\) and \(a(\overrightarrow{v} + \overrightarrow{w}) = a \overrightarrow{v} + b \overrightarrow{w}\)
\item \(- \overrightarrow{v} + \overrightarrow{v} = 0\)
\item \(0 + \overrightarrow{v} = \overrightarrow{v}\)
\end{itemize}

\begin{theorem}
0 in a vectorspace is unique.
\end{theorem}
\begin{proof}
Suppose \(0,0' \in V\), such that \(0 + \overrightarrow{v} = \overrightarrow{v}\) and \(0' + \overrightarrow{v} = \overrightarrow{v}\)

Then \(0 + \overrightarrow{v} = 0' + \overrightarrow{v} \longrightarrow 0 = 0'\)
\end{proof}

\subsection{Subspace}
\label{sec:orgheadline2}
\begin{definition}
\textbf{subspace}

Let \(W,V\) be vectorspaces.  If \(W \subseteq V\), then \(W\) is a subspace.

Equivalently, \(W\) is a subspace \(W \subseteq V\) and:
\begin{enumerate}
\item \(0 \in W\)
\item \(\overrightarrow{v} + \overrightarrow{w} \in W\) for \(\overrightarrow{v},\overrightarrow{w} \in W\)
\item \(a \overrightarrow{v} \in W\) for \(\overrightarrow{v} \in W, a \in \mathbb{R}\)
\end{enumerate}
\end{definition}

\begin{theorem}
The intersection of subspaces is a subspace.
\end{theorem}
\begin{proof}
Let \(W_1, W_2\) be subspaces of \(V\).  Show \(W_1 \cap W_2\) is also a subspace of \(V\)
\begin{enumerate}
\item \(0 \in W_1, W_2\), so \(0 \in W_1 \cap W_2\)
\item Let \(\overrightarrow{v}, \overrightarrow{w} \in W_1 \cap W_2\).  Then \(\overrightarrow{v}, \overrightarrow{w} \in W_1, W_2\).  By property of closed addition, \(\overrightarrow{v} + \overrightarrow{w} \in W_1 \cap W_2\)
\item Let \(a \in \mathbb{R}\), \(\overrightarrow{v} \in W_1 \cap W_2\).  Then \(\overrightarrow{v} \in W_1, W_2\).  By property of closed scalar multiplication, \(a \overrightarrow{w} \in W_1 \cap W_2\)
\end{enumerate}
\end{proof}

\begin{theorem}
Sum of vectorspaces is also a vectorspace.

\(W_1 + W_2 \defined \{w_1 + w_2 | w_1 \in W_1, w_2 \in W_2\}\)
\end{theorem}
\begin{proof}
Let \(W_1, W_2\) be subspaces of \(V\).  Show \(W_1 + W_2\) is also a vectorspace.
\begin{enumerate}
\item \(0 \in W_1, W_2\), so \(0 \in W_1 + W_2\)
\item Let \(\overrightarrow{v}, \overrightarrow{w} \in W_1 + W_2\), 

where \(\overrightarrow{v} = \overrightarrow{v_1} + \overrightarrow{v_2}\), \(\overrightarrow{w} = \overrightarrow{w_1} + \overrightarrow{w_2}\), \(\overrightarrow{v_1}, \overrightarrow{w_1} \in W_1\), \(\overrightarrow{v_2, \overrightarrow{w_2} \in W_2\)

Then \(\overrightarrow{v} + \overrightarrow{w} = \overrightarrow{v_1} + \overrightarrow{w_1} + \overrightarrow{v_2} + \overrightarrow{w_2}\).

Since \(\overrightarrow{v_1} + \overrightarrow{w_1} \in W_1\) and \(\overrightarrow{v_2} + \overrightarrow{w_2} \in W_2\),

then \(\overrightarrow{v} + \overrightarrow{w} \in W_1 + W_2\)
\item \(a \overrightarrow{v} = a \overrightarrow{v_1} + a \overrightarrow{v_2}\)

Since \(a \overrightarrow{v_1} \in W_1\) and \(a \overrightarrow{v_2} \in W_2\), \(a \overrightarrow{v} \in W_1 + W_2\)
\end{enumerate}
\end{proof}

\section{Span and Linear Dependence}
\label{sec:orgheadline5}
\begin{definition}
\textbf{span}

The span of a set \(S\) is all possible linear combinations of the set.

Let \(S = {v_1, ..., v_n}\)


\(\text{span} (S) = \{c_1 v_1 + ... + c_n v_n | c_1,...,c_n \in \mathbb{R}\}\)
\end{definition}

\begin{theorem}
Let \(S\) be a set, \(W\) be a vectorspace.

If \(S \subseteq W\), then \(\text{span} (S) \subseteq W\)
\end{theorem}
\begin{proof}
Let \(\overrightarrow{v} = c_1 s_1 + ... + c_n s_n \in \text{span} (S)\)

Since \(S\) is in \(W\), by the properties of closed addition and scalar multiplication, \(c_1 s_1 + ... + c_n s_n\) is also in \(W\)
\end{proof}

\begin{theorem}
The span of subspaces \(W_1 \cup W_2 = W_1 + W_2\).
\end{theorem}
\begin{proof}
-> Since \(W_1 \cup W_2 \subseteq W_1 + W_2\), \(\text{span} (W_1 \cup W_2) \subseteq W_1 + W_2\) (since vectorspace containing set \(S\) also contains \(\text{span} (S)\)

<- Let \(\overrightarrow{v} \in W_1\), \(\overrightarrow{w} \in W_2\)

Since \(\overrightarrow{v}, \overrightarrow{w} \in W_1 \cup W_2\), 

\(\overrightarrow{v} + \overrightarrow{w} \in W_1 \cup W_2\) and \(\overrightarrow{v} + \overrightarrow{w} \in \text{span} (W_1 \cup W_2)\)

therefore \(W_1 + W_2 \subseteq \text{span} (W_1 \cup W_2)\)
\end{proof}

\begin{definition}
\textbf{linear independence}

A set \(S = \{s_1, ..., s_n\}\) is linearly independent if \(c_1, ..., c_n = 0\) is the only solution to \(c_1 s_1 + ... + c_n s_n = 0\)

Equivalently, a set is linearly independent if every element cannot be expressed as a combination of the other elements.
\end{definition}

\begin{examples}
\begin{enumerate}
\item Is the set \(S = \{1 + x + x^2 + x^3, x + x^2 + x^3, x^2 + x^3, x^3\}\) linearly independent?

\(c_1(1 + x + x^2 + x^3) + c_2 ( x + x^2 + x^3) + c_3 ( x^2 + x^3) + c_4 x^3 = 0\)
\(c_1 + (c_1 + c_2)x + (c_1 + c_2 + c_3)x^2 + (c_1 + c_2 + c_3 + c_4)x^3 = 0\)
\(c_1 = 0\)
\((c_1 + c_2)x = 0\)
\((c_1 + c_2 + c_3)x^2 = 0\)
\((c_1 + c_2 + c_3 + c_4)x^3 = 0\)

\(c_1 = c_2 = c_3 = c_4 = 0\)
\end{enumerate}
\end{examples}

\subsection{Subset Dependence}
\label{sec:orgheadline4}
\begin{theorem}
If \(S_1 \subseteq S_2\) and \(S_1\) is dependent, then \(S_2\) is dependent.
\end{theorem}
\begin{proof}
Let \(S_1 = \{\overrightarrow{v_1}, ...,\overrightarrow{v_m}\}\) be dependent and \(S_1 \subseteq S_2\)

Then at least one \(c\) is nonzero in \(c_1 \overrightarrow{v_1} + ... + c_m \overrightarrow{v_m} = 0\).

Then at least one \(c\) is nonzero in \(c_1 \overrightarrow{v_1} + ... + c_m \overrightarrow{v_m} + ... + c_n \overrightarrow{n} = 0\).

so \(S_2\) is linearly dependent.
\end{proof}

\begin{theorem}
If \(S_1 \subseteq S_2\) and \(S_2\) is independent, then \(S_1\) is also independent.
\end{theorem}
\begin{proof}
Let \(S_2 = \{\overrightarrow{v_1}, ...,\overrightarrow{v_n}\}\) be independent and \(S_1 \subseteq S_2\)

Then all \(c\) must be zero in \(c_1 \overrightarrow{v_1} + ... + ... + c_m \overrightarrow{v_m} + ... + c_n \overrightarrow{v_n} = 0\).

Then all \(c\) must be zero in \(c_1 \overrightarrow{v_1} + ... + c_m \overrightarrow{v_m} = 0\).

so \(S_1\) is linearly independent.
\end{proof}

\section{Basis and Dimension}
\label{sec:orgheadline6}
\begin{definition}
A set \(B\) is called a basis of vectorspace \(V\) if

\begin{itemize}
\item it is linearly independent and a generating set (\(V \subseteq \text{span} (B)\))

or

\item \(B\) is the smallest set of \(V\) such that \(\text{span} (B) = V\)

or

\item every vector in \(V\) can be expressed uniquely from linear combinations of \(B\)
\end{itemize}
\end{definition}
\begin{proof}
Show that every vector in \(V\) can be expressed uniquely in terms of the basis \(B\)

Let \(\beta\) be a basis of \(V\) and \(\overrightarrow{V} \in V\)

\(\overrightarrow{v} = c_1 b_1 + ... + c_n b_n, \overrightarrow{v} = c_1' b_1 + ... c_n' b_n\) for \(c_1, ..., c_n, c_1', ..., c_n'\)

then  \(c_1 b_1 + ... + c_n b_n = c_1' b_1 + ... + c_n' b_n\)

\(0 = (c_1 - c_1')b_1 + ... + (c_n - c_n')b_n\)

Since \(B\) is linearly independent, \(c_1 = c_1', ..., c_n = c_n'\).

Each vector is uniquely expressed in terms of \(B\)
\end{proof}

\begin{theorem}
Let \(L\) be an independent set of \(V\) with \(m\) elements.
\textbf{An indepenent set can be extended to create a generating set}
\end{theorem}

\begin{theorem}
Every basis of a finite vectorspace has the same number of elements.
\end{theorem}
\begin{proof}
Let \(B_1, B_2\) be bases of \(V\) where \(\dim (B_1) = n, \dim(B_2) = m\)

Since \(B_1\) is a generating set and \(B_2\) is an independent set, \(|B_1| \geq |B_2|\)

Since \(B_2\) is a generating set and \(B_1\) is an independent set, \(|B_2| \geq |B_1|\)

So \(|B_1| = |B_2|\)
\end{proof}

\begin{theorem}
For every vectorspace of dimension \(n\)

\begin{enumerate}
\item \(\dim(\text{generating set}) \geq n\)

\item \(\dim(\text{independent set}) \leq n\)
\end{enumerate}
\end{theorem}
\begin{proof}
\begin{enumerate}
\item Let \(G\) be a generating set and \(B\) be a basis of some vectorspace of dimension \(n\)

Since \(G\) is a generating set and \(B\) is independent, \(|G| \geq |B| = n\)

\item Let \(L\) be a linearly independent set and \(B\) be a basis of some vectorspace of dimension \(n\)

Since \(L\) is a linearly independent set and \(B\) is a generating set, \(|L| \leq |B| = n\)
\end{enumerate}
\end{proof}

\begin{theorem}
For a vectorspace \(V\) of dimension \(n\)

\begin{enumerate}
\item A generating set with \(n\) elements is a basis of \(V\)

\item An independent set with \(n\) elements is a basis of \(V\)
\end{enumerate}
\end{theorem}
\begin{proof}

\end{proof}

\begin{theorem}
For a vectorspace \(V\) of dimension \(n\)

\begin{enumerate}
\item A generating set with dimension greater than \(n\) can be reduced to be a basis.

\item An independent set with dimension less than \(n\) can be extended to be a basis.
\end{enumerate}
\end{theorem}
\begin{proof}

\end{proof}

\section{Dimension of Subspace}
\label{sec:orgheadline10}
\begin{theorem}
Let \(W\) be a subspace of \(V\).  If the dimensions of \(W\) equals \(V\), then \(V = W\).  If the dimension of \(W\) is zero, then \(W = \{0\}\)

\textbf{Corollary}

A basis \(B\) of \(W\) can be extended to be a basis of \(V\).
\end{theorem}

\subsection{Properties}
\label{sec:orgheadline9}
\subsubsection{Sum}
\label{sec:orgheadline7}

\(\dim(W_1 + W_2) = \dim (W_1) + \dim (W_2) - \dim (W_1 \cap W_2)\)

\begin{examples}
\begin{enumerate}
\item Line and coincident plane

\includegraphics[width=.9\linewidth]{./subspace_dim1.png} 

\(\dim(\text{line + plane}) = 2 = \dim(\text{line}) + \dim(\text{plane}) - \text{dim(line} \cap \text{plane)}\)

where \(\text{dim(plane)} - \text{dim(line} \cap \text{plane)} = 1\)

\item Line and noncoincident plane

\includegraphics[width=.9\linewidth]{./subspace_dim2.png} 

\(\dim(\text{line + plane}) = 3 = \dim(\text{line}) + \dim(\text{plane}) - \dim(\text{line} \cap \text{plane)}\)

where \(\dim(\text{plane}) - \dim(\text{line} \cap \text{plane)} = 0\)
\end{enumerate}
\end{examples}

\subsubsection{Direct Sum}
\label{sec:orgheadline8}

\(W_1 + W_2 = W_1 \oplus W_2\) iff \(\text{dim} (W_1 + W_2) = \text{dim} (W_1) + \text{dim} (W_2)\)

\begin{proof}
\(\Rightarrow\) Assume \(W_1 + W_2 = W_1 \oplus W_2\).

Then \(W_1 \cap W_2 = \{0\}\), so \(\text{dim} (W_1 \cap W_2) = 0\)

Then by the sum property, \(\text{dim} (W_1 + W_2) = \text{dim} (W_1) + \text{dim} (W_2)\)

\(\Leftarrow\) Assume \(\text{dim} ( W_1 + W_2) = \text{dim} (W_1) + \text{dim} (W_2)\)

Then \(\text{dim} (W_1 \cap W_2) = 0\), so \(W_1 \cap W_2 = \{0\}\)

therefore \(W_1 + W_2 = W_1 \oplus W_2\)
\end{proof}

\section{Linear Transformations}
\label{sec:orgheadline12}
\begin{definition}
A linear transformation is a mapping \(T:V \rightarrow W\) from one vectorspace to another such that

\begin{itemize}
\item addition is preserved

\begin{itemize}
\item Let \(a,b \in V\).  \(T(a) + T(b) = T(a + b)\)
\end{itemize}

\item scalar multiplication is preserved

\begin{itemize}
\item Let \(a \in \mathbb{R}, x \in V\), \(T(ax) = aT(x)\)
\end{itemize}
\end{itemize}
\end{definition}

\subsection{Properties}
\label{sec:orgheadline11}
\begin{itemize}
\item \(T(\overrightarrow{0}) = \overrightarrow{0}\)
\begin{itemize}
\item proof: \(T(\overrightarrow{0}) = T(0*\over{v}) = 0T(\overrightarrow{v}) = \overrightarrow{0}\)
\end{itemize}
\item \(T(\overrightarrow{u} - \overrightarrow{v}) = T(\overrightarrow{u}) - T(\overrightarrow{v})\)
\begin{itemize}
\item proof: \(T(\overrightarrow{u} - \overrightarrow{v}) = T(\overrightarrow{u}) + T(-1 \overrightarrow{v}) = T(\overrightarrow{u}) - T(\overrightarrow{v})\)
\end{itemize}
\end{itemize}

\begin{examples}
\begin{enumerate}
\item For \(T:P_n(\mathbb{R}) \rightarrow P_{n-1}(\mathbb{R})\) where \(T(f) = f'\) for \(f \in P_n(\mathbb{R})\).  Show \(T\) is linear.

Let \(a,b \in \mathbb{R}\), \(f,g \in P_n(\mathbb{R})\).

\(T(af + bg) = (af + bg)' = af' + bg' = aT(f) + bT(g)\)
\end{enumerate}
\end{examples}

\begin{definition}
\textbf{Identity Transformation}

A mapping \(T:v \rightarrow W\) such that \(T(\overrightarrow{v} = \overrightarrow{v}\) for \(\overrightarrow{v} \in V\)
\#end\(_{\text{definition}}\)

\#+begin\(_{\text{definition}}\)
\textbf{Zero Transformation}

A mapping \(T:V \rightarrow W\) such that \(T(\overrightarrow{v}) = \overrightarrow{0}\) for \(\overrightarrow{v} in V\)
\end{definition}

\section{Kernel and Image}
\label{sec:orgheadline13}
\begin{definition}
\textbf{Kernel}

The kernel (\(\ker(T)\) or \(N(T)\)) of a transformation \(T:V \rightarrow W\) is the set of vectors in \(V\) that map to \(\overrightarrow{0}\) in \(W\)

\includegraphics[width=.9\linewidth]{./kernel.png}
\end{definition}

\begin{definition}
\textbf{Image}

The image (\(\ima(T)\) or \(R(T)\)) of a transformation \(T:V \rightarrow W\) is the set of vectors in \(W\) mapped to by vecotrs in \(V\).

\includegraphics[width=.9\linewidth]{./image.png}
\end{definition}

\begin{examples}
\begin{enumerate}
\item Prove the kernel of \(T\) is a subspace of \(V\). (show \(0 \in \ker(T)\), \(\ker(T)\) closed for addition, \(\ker(T)\) closed for scalar multiplication
\begin{enumerate}
\item By definition \(T(0) = 0\), so \(0 \in \ker(T)\)
\item For \(\overrightarrow{u}, \overrightarrow{v} \in \ker(T)\), we have \(T(\overrightarrow{u}) = T(\overrightarrow{v}) = 0\)

\(T(\overrightarrow{u} + \overrightarrow{v}) = T(\overrightarrow{u}) + T(\overrightarrow{v}) = 0\)

so \(\overrightarrow{u} + \overrightarrow{v} \in \ker(T)\)
\item For \(a \in R\), \(v \in \ker(T)\)  we have \(T(\overrightarrow{v}) = 0\)

\(T(a\overrightarrow{v}) = aT(\overrightarrow{v}) = 0\), so \(a \overrightarrow{v} \in \ker(T)\)
\end{enumerate}
\item Let \(T:\mathbb{R}_3 \to \mathbb{R}_2\), where \(T \left( \begin{matrix} a\\b\\c \end{matrix} \right) = \left( \begin{matrix} a-b \\ 0 \\ 2c \end{matrix} \right)\). Find the kernel of \(T\)

\(\ker(T) = \Set{\left( \begin{matrix} a\\b\\c \end{matrix} \right) | T \left( \begin{matrix} a\\b\\c \end{matrix} \right) = 0 }\)

\(\left( \begin{matrix} a-b \\ 0 \\ 2c \end{matrix} \right) = 0\), so \(a = b\) and \(c = 0\)

\(\ker(T) = \Set{\left( \begin{matrix} a\\a\\0 \end{matrix} \right) | a \in \mathbb{R} }\)
\item Find the image of \(T\).

\(\ima(T) = \Set{ T(\overrightarrow{v}) | \overrightarrow{v} \in \mathbb{R}_3 } = \Set{ \left( \begin{matrix} a-b \\ 0 \\ 2c \end{matrix} \right) | a,b,c \in \mathbb{R} } = \Set{ \left( \begin{matrix} a\\ 0 \\ b \end{matrix} \right) | a,b \in \mathbb{R} }\)
\end{enumerate}
\end{examples}
\end{document}
